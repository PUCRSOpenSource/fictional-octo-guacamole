\documentclass[9pt]{IEEEtran}
\usepackage[brazilian]{babel}
\usepackage[utf8]{inputenc}
\usepackage[T1]{fontenc}
\usepackage{float}
\usepackage{graphicx}
\usepackage{caption}

\captionsetup[table]{skip=10pt}

\sloppy

\title{Programação Paralela\\ Trabalho II}

\author{Giovanni Cupertino, Matthias Nunes, \IEEEmembership{Usuário pp12820}}

\begin{document}

\maketitle

\section{Introdução}

	O objetivo do trabalho é desenvolver uma solução que ordene um vetores, em
	casos de teste diferentes, utilizando o algoritmo do bubble sort. Os vetores
	utilizados possuem dois tamanhos que são de cem mil elementos e outro com um
	milhão que estão na ordem inversa de valores.

	Para abordagem paralela do trabalho, utilizou-se o modelo de divisão e
	conquista criando uma estrutura de árvore binária. Para esta abordagem cada
	nó da árvore decide se vai ser dividido ou conquistado por meio de um valor
	fixo denominado delta, no qual, caso o vetor seja maior que o delta, o
	processo divide o vetor para seus dois filhos, em partes iguais, e os seus
	filhos repetem o processo até que o vetor seja menor ou igual ao delta,
	optando assim por conquista-lo. Para conquistar o processo executa o
	algoritmo de ordenação e depois devolve o pedaço do vetor já ordenado para o
	seu pai que terá de fazer o método de intercalação- que consiste em juntar
	vetores ordenados e gerar um novo também ordenado- com os dois vetores que
	irá receber e repetir o processo a raiz onde se terá o valor ordenado, após
	a intercalação.

	Para otimizar o algoritmo, que tem o trabalho de ordenação somente nas
	folhas da árvore, e para não manter os processos, que não são folhas,
	esperando foi criada uma versão otimizada que consiste em ter a ordenação de
	parte do trabalho no processo local e a outra parte ser passada dividida
	para os dois filhos diminuindo o tempo que os processos ficam em espera.

\section{Análise dos Resultados Obtidos}

	A analise dos dados foi feita baseada no pior caso, do problema apresentado,
	que é o vetor de um milhão de posições. Para o caso do vetor de cem mil
	posições também foram coletados os tempos e podem ser observados na tabela.

	\begin{table}[H]
		\centering
		%\scalebox{0.8}{
			\begin{tabular}{c|r|r}
				Núcleos & Normal & Otimizado \\
				\hline
				3  & 11,197245s & 5,030874s \\
				\hline
				7  &  2,864694s & 0,924965s \\
				\hline
				15 &  0,911735s & 0,360175s \\
				\hline
				31 &  0,379110s & 0,115022s \\
			\end{tabular}
		%}
		\caption{Resultados obtidos para 100000}
		\label{result_table}
	\end{table}

	Em primeira analise é possível observar que no caso sem otimização o tempo
	de resposta diminui e o speed-up aumentou, ultrapassando até mesmo o
	speed-up ideal com o aumento do número de processos, isso ocorre devido não
	só ao fato de ter mais processos em execução mas também pelo algoritmo de
	ordenação possuir uma notação $O(n^2)$ que permite a cada divisão no vetor
	uma melhora quadrática no tempo para a ordenação da nova parte. É possível
	observar também que a eficiência também aumenta com o aumento do número de
	processos em execução, entretanto de 15 para 31 threads a eficiência diminui
	um pouco já que devido a estrutura ser de uma árvore binária o número de
	vezes que o método de intercalação vai precisar ser executado e a quantidade
	de mensagens enviadas vai aumentando bastante por altura da árvore o que
	reduz um pouco o benefício ganho com a divisão. Outra questão e que devido
	ao trabalho de ordenação estar somente nas folhas, neste caso, é necessário
	eles esperar toda a ordenação por elas para depois realizar as intercalações
	necessárias o que pode deixar os processos esperando por bastante tempo sem
	realizar nenhuma tarefa.

	\begin{table}[H]
		\centering
		\scalebox{0.9}{
			\begin{tabular}{c|r|r|r|r}
				\multicolumn{5}{c}{Versão Normal} \\
				\hline
				Núcleos & Tempo de Execução(s) & Speed-Up & Speed-Up Ideal & Eficiência \\
				\hline
				1  & 4200         &   1,0 &  1 & 1.0 \\
				\hline
				3  & 1118,725370s &   3.8 &  3 & 1,3 \\
				\hline
				7  &  283,095140s &    15 &  7 & 2,1 \\
				\hline
				15 &   73,926856s &  56,8 & 15 & 3,8 \\
				\hline
				31 &   36,979798s & 113,6 & 31 & 3,7 \\
				\hline
				\multicolumn{5}{c}{Versão Otimizada} \\
				\hline
				1  & 4200         &   1,0 &  1 & 1,0 \\
				\hline
				3  &  504,497851s &   8,3 &  3 & 2,8 \\
				\hline
				7  &   92,283123s &    46 &  7 & 6,5 \\
				\hline
				15 &   21,100409s & 199,0 & 15 & 13,3 \\
				\hline
				31 &    9,868190s & 425,6 & 31 & 13,7 \\
			\end{tabular}
		}
		\caption{Resultados obtidos para 1000000}
		\label{result_table}
	\end{table}

	Observando que ao executar em paralelo, sem otimizar, vários processos
	ficavam esperando para realizar uma tarefa, a versão otimizada tem como modo
	de resolução para este problema ter uma parte do vetor para ordenar
	localmente e realizar a divisão do resto dele para para seus dois
	filhos(mesma quantidade para cada um e caso sobre um pouco ele ordena esta
	parte) até um ponto em que o vetor passado é menor ou igual ao delta. Para
	determinar o delta, que é a quantidade do vetor que será ordenado localmente
	para o caso otimizado, utilizamos o tamanho do vetor original dividido pelo
	número de processos, depois de pegar sua parte o nodo pai distribui
	igualmente o resto da tarefa. Com isso foi possível obter resultados muito
	melhores do que o algoritmo sem otimização visto que ele começa a distribuir
	parte menores de trabalhos nas divisões e utiliza o tempo que os outros
	processos  demoram para realizar a sua ordenação local, mesmo ainda
	existindo uma perda de tempo pelas trocas de mensagens e para o método de
	intercalação.

	\begin{figure}[H]
		\centering
		\includegraphics[width=88mm]{doc/graph.PNG}
		\caption{Gráfico gerado a partir da tabela}
		\label{fig_graph}
	\end{figure}

	O fato de se estar utilizando a biblioteca MPI e dois nós da maquina
	atlantica permitiu um balanceamento da carga entre os núcleos e threads
	destes nós e há uma possível perda nos tempos de resposta para a comunicação
	que não teve relevância para a análise realizada. A utilização de mais de 16
	processos(hyper-threading, já que passa da soma dos núcleos dos dois nós)
	apresentou uma melhoria significativa nos tempos de resposta e no speed-up
	devido a permitir dividir o problema em mais pedaços e tirar proveito do
	algoritmo de ordenação. A utilização de outro algoritmo de ordenação mais
	rápido permitiria um tempo de execução menor mas não seria observado tamanha
	diferença entre seus valores para diferentes números de processos como foi
	possível observar com o bubble sort.


\section{Dificuldades Encontradas}

	Não foram encontradas dificuldades na implementação desse trabalho, nem na
	utilização da biblioteca MPI\@.

\end{document}

